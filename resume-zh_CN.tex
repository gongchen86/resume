% !TEX TS-program = xelatex
% !TEX encoding = UTF-8 Unicode
% !Mode:: "TeX:UTF-8"

\documentclass{resume}
\usepackage{zh_CN-Adobefonts_external} % Simplified Chinese Support using external fonts (./fonts/zh_CN-Adobe/)
%\usepackage{zh_CN-Adobefonts_internal} % Simplified Chinese Support using system fonts
\usepackage{linespacing_fix} % disable extra space before next section
\usepackage{cite}

\begin{document}
\pagenumbering{gobble} % suppress displaying page number

\name{龚辰}

\basicInfo{
  \email{gongchen@bit.edu.cn} \textperiodcentered\
  \phone{(+86) 156-5296-7528} \textperiodcentered\
  \linkedin[gong-chen]{https://cn.linkedin.com/in/gong-chen}}

\section{\faGraduationCap\  教育背景}
\datedsubsection{\textbf{专用处理器研究所,北京理工大学}, 北京}{2012 -- 至今}
\textit{博士研究生}\ 电子科学与技术, 预计 2018 年 7 月毕业

\textit{保研}\ 硕博连读期间导师:刘大可(“千人计划”,专用处理器领域的国家特聘专家)
\datedsubsection{\textbf{大连理工大学}, 辽宁-大连}{2008 -- 2012}
\textit{学士}\ 电子科学与技术,\textit{辅修}\ 计算机科学与技术,\textit{创新实践}\ 机电创新实践基地-智能车研究室

\section{\faUsers\ 项目经历}
\datedsubsection{\textbf{微小尺寸体内植入式医疗微系统}}{2015 年1 月 -- 至今}
\role{国家重大科研仪器研制专项}{核心研究人员}
\textbf{项目描述:}设计一个植入眼内的微系统,用于难治性青光眼治疗(青光眼是一种由于眼内房水液压过高而致盲的难治性眼病)。该系统基于无线信能同传、微传感器和微泵,实现眼内压自动检测和调控。

\textbf{本人职责:}负责\underline{体内外通信系统}及\underline{体内微控制处理器}的设计、优化与实现。主要解决超低耦合下的可靠近场通信,体内超低功耗接收机设计等植入式医疗设备研制的共性问题。
\begin{itemize}
  \item 提出了一种寄生在无线供能上的高可靠近场通信方案
  \item 提出了一种体内超低开销BPSK解调器
  \item 提出了一种无晶振高精度的时钟源(全数字锁频环,模数混合设计)
  \item 设计了体内微控制处理器,编写指令集文档,时钟精确的模拟器实现,以及完成模拟器和指令集的一致性测试和拐点测试。
\end{itemize}

\datedsubsection{\textbf{SIMD并行存储访问无冲突优化工具}}{2014 年6 月 -- 2015年5月}
\role{国家863重大专项子课题}{核心研究人员}
为高灵活度前向纠错处理器设计的SIMD指令并行存储访问无冲突优化工具。该处理器是基于多片ScratchPad Memory的SIMD架构的处理器,需要解决并行化LDPC和Turbo纠错算法做交织操作时遇到的存储访问冲突的问题。
%\begin{onehalfspacing}
\begin{itemize}
  \item 使用无向连接图对并行存储数据相关性建模
  \item 提出内存平衡式的最大度启发式图染色算法(Balanced DSATUR Coloring Algorithm)
  \item 可在片下(编译时)得到可预测性算法的数据安排表(Data Allocation Table)和并行访存的寻址表和交织表(Addressing Table, Permutation Table)
\end{itemize}
%\end{onehalfspacing}

\datedsubsection{\textbf{专用处理器加速自动扩展工具}}{2012 年11 月 -- 2014年5月}
\role{实验室自研项目}{核心研究人员}
%\begin{onehalfspacing}
基于LLVM做了对特定应用域ASIP的自动指令集扩展,当时主要基于LLVM 中间表示的优化过程,通过自定义的pass,对testbench做静态和动态的程序分析,得到热点kernel的数据流图从而给出加速指令的扩展建议。这个工作是我们实验室专用处理器高层次综合工具NoGap的一个子课题。
\begin{itemize}
  \item 熟悉了Linux开发环境和项目管理,Bash Shell, Make, CMake.
  \item 学习了编译原理和LLVM基本知识。
  \item 实践了程序静态分析和动态分析。
\end{itemize}
%\end{onehalfspacing}

\datedsubsection{\textbf{高能效SIMD处理器设计}}{2014 年3 月 -- 2014 年11 月}
\role{实验室自研项目}{主要研究人员}
%\begin{onehalfspacing}
\textbf{项目描述:}设计一个面向未来基带、多媒体和3D游戏等基于确定性算法应用的高能效处理器。处理器利用并行数据通路和具有多种模式的寻址单元进行高效的并行计算,其主要特性包括:
\begin{itemize}
  \item 8路并行的定点SIMD 数据通路,可在一个时钟内完成8个实数乘加操作或者4个复数乘加操作。
  \item 片上数据便笺存储器可分为8个块,依靠数据交织完成无冲突存取。
  \item 具有SIMD、操作融合、魔术指令三种类型的加速指令。
\end{itemize}

\textbf{本人职责:}负责DCT扩展指令设计、SIMD数据
\begin{itemize}
  \item 熟悉了Linux开发环境和项目管理,Bash Shell, Make, CMake.
  \item 学习了编译原理和LLVM基本知识。
  \item 实践了程序静态分析和动态分析。
\end{itemize}

% Reference Test
%\datedsubsection{\textbf{Paper Title\cite{zaharia2012resilient}}}{May. 2015}
%An xxx optimized for xxx\cite{verma2015large}
%\begin{itemize}
%  \item main contribution
%\end{itemize}

\section{\faCogs\ IT 技能}
% increase linespacing [parsep=0.5ex]
\begin{itemize}[parsep=0.5ex]
  \item 编程语言: C == Python > C++ > Java
  \item 平台: Linux
  \item 开发: xxx
\end{itemize}

\section{\faHeartO\ 获奖情况}
\datedline{\textit{第一名}, xxx 比赛}{2013 年6 月}
\datedline{其他奖项}{2015}

\section{\faInfo\ 其他}
% increase linespacing [parsep=0.5ex]
\begin{itemize}[parsep=0.5ex]
  \item 技术博客: http://blog.yours.me
  \item GitHub: https://github.com/username
  \item 语言: 英语 - 熟练(TOEFL xxx)
\end{itemize}

%% Reference
%\newpage
%\bibliographystyle{IEEETran}
%\bibliography{mycite}
\end{document}
